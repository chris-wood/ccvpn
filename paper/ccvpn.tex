\documentclass[conference,letterpaper,10pt]{IEEEtran}
\usepackage{blindtext, graphicx, url, paralist}

% Add the compsoc option for Computer Society conferences.
%
% If IEEEtran.cls has not been installed into the LaTeX system files,
% manually specify the path to it like:
% \documentclass[conference]{../sty/IEEEtran}

% *** GRAPHICS RELATED PACKAGES ***
%
\ifCLASSINFOpdf
  % \usepackage[pdftex]{graphicx}
  % declare the path(s) where your graphic files are
  % \graphicspath{{../pdf/}{../jpeg/}}
  % and their extensions so you won't have to specify these with
  % every instance of \includegraphics
  % \DeclareGraphicsExtensions{.pdf,.jpeg,.png}
\else
  % or other class option (dvipsone, dvipdf, if not using dvips). graphicx
  % will default to the driver specified in the system graphics.cfg if no
  % driver is specified.
  % \usepackage[dvips]{graphicx}
  % declare the path(s) where your graphic files are
  % \graphicspath{{../eps/}}
  % and their extensions so you won't have to specify these with
  % every instance of \includegraphics
  % \DeclareGraphicsExtensions{.eps}
\fi

\usepackage{microtype}

\usepackage[tight,footnotesize]{subfigure}
\usepackage{fixltx2e}

\usepackage{multirow}

\usepackage{algorithm2e}
\usepackage{algorithm,algorithmic}
\usepackage{xcolor}
\newcommand\todo[1]{\textcolor{red}{#1}}

\hyphenation{op-tical net-works semi-conduc-tor}

\newtheorem{definition}{\textbf{Definition}}[section]
\newtheorem{theorem}{\textbf{Claim}}[section]
\newtheorem{corollary}{\textbf{Corollary}}[theorem]
\newtheorem{lemma}[theorem]{\textbf{Lemma}}

\begin{document}

\title{Tunnels to Towers: Secure Virtual Private Networking for CCN}


\maketitle


\begin{abstract}
Content Centric Networking (CCN) is a future Internet architecture which is envisioned as an alternative to the current IP-based model. CCN emphasizes content distribution by making contents directly addressable in a request-based information centric network. An advantage of CCN is that it has some innate privacy friendly features, such as lack of source and destination addresses in packets. However, to be considered a viable future Internet architecture, CCN must offer services for private and anonymous communication that are at least equivalent for  to those present in the IP architecture. Among such, VPNs are a very popular application that enables users to send and receive data across shared or public networks as if their computing devices were directly connected to the same private network. In this work we design, implement and evaluate CCVPN, a content centric virtual private network capable of providing the same functionality of VPNs within the CCN Internet architecture. To the best of our knowledge CCVPN is the first proposal of a VPN alike service for CCNs. In addition to the CCVPN design, we also provide a security analysis and experimental performance evaluation for this new system.
\end{abstract}

\begin{IEEEkeywords}
%Human Mobility, Group Detection, Group Dynamics, Periodicity, Opportunistic Routing.
\end{IEEEkeywords}

\IEEEpeerreviewmaketitle

\section{Introduction}
Content-centric networking (CCN) is a type of request-based information-centric
networking (ICN) architecture. In CCN, all data is named. Consumers obtain data
by issuing an explicit request for the content by its name. The network is
responsible for forwarding this request towards producers, based on the name,
who then generate and return the content response. Since a name uniquely identifies
a content response, routers may cache these packets to use in response to
future requests for the same name. As a consequence, all content has an (implicit
or explicit) authenticator that is used to verify the name-to-data binding. In
order to prevent cache poisoning attachs, wherein a malicious producer supplies
fake data in a content response that is propogated in the network, a router
should never serve (a) content with an invalid authenticator or (b) cached
content that it cannot verify. To enable (b), content objects with a digital
signature are expected to carry the public verification key or certificate. If the
authenticator is a MAC, then intermediate routers cannot verify it and should
therefore not cache the content.

One negative side effect of name-based requests is that any on-path or 
eavesdropping adversary between a consumer and producer can learn the identity
and contents of all data in transit. In traditional IP-based networks, there are
generally two types of mechanisms to solve this problem: (1) anonymity
networks such as Tor \cite{tor} or (2) VPNs. As tools focused on anonymity, 
the former help prevent linkability of packets to their requestors without 
always protecting the identities or content themselves. In contrast, VPNs focus
on packet confidentiality by creating a tunnel between two private networks
or a consumer and single private network. All traffic over this tunnel is
encrypted and thereby opaque to an eavesdropper. VPNs differ from anonymity
networks such as Tor in that they are \emph{network-layer} mechanisms that
typically only introduces a single layer of encryption to protect traffic.
Thus, while Tor can be used to enable VPN-like functionality, it is often 
far more inefficient since it operates above the network layer.

In ICN (or more specifically, CCN and NDN), ANDaNA was the first anonymity network of its 
kind. Similar to Tor, ANDaNA uses circuits formed from anonymizing routers (ARs)
to marshall requests and responses between consumers and producers. The former
onion-encrypt interests and content using the public key(s) of the target ARs.
A variant of ANDaNA uses symmetric keys for packet encapsulation but suffers from
linkability. Tsudik et al. \cite{ac3n} proposed an optimized version of the symmetric-key
ANDaNA variant that did not permit linkability. To the best of our knowledge, there
is no anonymity network variant for ICN architectures. Though tunneling is only 
useful for only a subset of ICN traffic, we believe it is a gap to be addressed 
for this emerging technology, for a variety of reasons. {\bf First}, 

XXX: consumer privacy outside an AS is important, e.g., XXX
XXX: 
XXX: 




\input{02-overview}
\section{Related Work} \label{sec:related}

\begin{itemize}
\item \cite{tsudik2016ac3n}
\item \cite{dibenedetto2011andana}
\item ...
\end{itemize}

Content-based encryption is arguably the most popular technique for protecting CCN content from
unauthorized disclosure. This technique permits content to be disseminated throughout
the network since it cannot be decrypted by adversaries without the appropriate decryption key(s).
Many variations of this approach have been proposed based on general
group-based encryption \cite{Smetters2010}, broadcast encryption \cite{Misra2013,Ion2013} and
proxy re-encryption \cite{Wood2014}. Kurihara et al. \cite{ifip15} generalized these specialized
approaches in a framework called CCN-AC, an encryption-based access control framework
that shows how to use manifests to explicitly specify and enforce other encryption-based
access control policies. Consumers use information in the manifest to (1) request appropriate
decryption keys and (2) use them to decrypt content object(s). The NDN NBAC \cite{yu2015name}
scheme is similar to \cite{ifip15} in that it allows decryption keys to be
flexibly specified by a data owner. However, it does this based on name engineering rules instead of
configuration. Interest-based access control \cite{ghali2015interest} is a different
type of access control scheme wherein content was optionally encrypted. Access
was protected by making the names of content derivable by only authorized consumers.
NDN-ACE \cite{shangndn} is a recent access control framework for IoT environments
which includes a key exchange protocol for distributing secret keys to sensors.
We revisit NDN-ACE in Section \ref{sec:compare}.

\input{04-design}
\section{Security Analysis}\label{sec:sec-analysis}

\todo{TODO: Introductory text about the sec analysis}

\begin{theorem}
\textit{Let $Encapsulate_{pk}(I_p)$ denote the interest encapsulation routine described in Algorithm~\ref{alg:interestEncap}. Given any two interests $I_p^1$ and $I_p^2$ chosen by the adversary, if $Enc_{pk}$, used to construct $Encapsulate_{pk}$, is a CPA-secure public-key encryption scheme, then the adversary has only negligible advantage on determining a randomly selected bit $b$,  when $I_e^b = Encapsulate_{pk}(I_p^b)$ is given to the adversary.
}\end{theorem}

\textit{Proof--}
\todo{TODO}

\begin{theorem}
\textit{Let $ContentEnc_{k_r}(C_p)$ denote the content encryption routine described in Algorithm~\ref{alg:contentEnc}. Given any two contents $C_p^1$ and $C_p^2$ chosen by the adversary, if $EncryptThenMAC_{k_r}$, used to construct $ContentEnc_{sk}$, is a CCA-secure symmetric-key encryption scheme , then:
\begin{enumerate}
\item the adversary has only negligible advantage on determining a randomly selected bit $b$, when $C_e^b = ContentEnc_{k_r}(C_p^b)$ is given to the adversary;
\item the adversary has negligible probability of successfully forging an encrypted content $I_c'$.
\end{enumerate}
}\end{theorem}

\textit{Proof--}
\todo{TODO}

\input{06-perf-analysis}
\section{Experiments}

\subsection{Experimental Methodology}

In this section we empirically evaluate the CCVPN design paying special attention to the metrics that were earlier discussed in Sec.~\ref{sec:analysis}, i.e., processing overhead, network throughput, and state consumption. In our evaluation we consider the two versions of CCVPN: public key version, and symmetric key version. We here recall that the symmetric key version relies on the assumption that a secure key agreement protocol is performed between the domain's gateways prior to the CCVPN protocol execution.

Our testbed network consists of a butterfly topology, in which the consumers' side and the producers' side gateways are directly interconnected. $N$ producers are connected to the producers' domain gateway and $M$ consumers are connected to the consumers' domain gateway (see Fig.~\ref{create_figure}).

To investigate the processing overhead we measure the average time demand for computing the interests' encapsulation (in public and symmetric key versions), interest decapsulation, content encryption, and content decryption for different content packet sizes (1024, 4096, 16384, and 65536 bytes). We also measure the state consumption for these same four functions.

To compute the overall network throughput we measure the average data-rate for transmissions of 1 to 1,000,000 different interests issues per consumer. We also vary the number of consumers and producers from 1 to 10 of each. Finally, in addition to the network throughput, we also exhibit the total transmission delay for each of the experiments.

\subsection{Results}

\todo{TODO: Improve result figures (less figures more results in same figure)}

\begin{table}[!h]
\centering
\caption{Interest encapsulation processing times}
\label{my-label}
\begin{tabular}{|l|l|l|l|l|}
\hline
Encapsulation mode   & Encapsulation & Decapsulation \\ \hline
Public Key  & 444$\mu s$           & 449$\mu s$           \\ \hline \cline{4-5} 
Symmetric Key & TODO          & TODO          \\ \hline
\end{tabular}
\end{table}



\begin{table}[!h]
\centering
\caption{Content encryption and decryption times for different payload sizes}
\label{my-label}
\begin{tabular}{|l|l|l|}
\hline
Packet size        & Encryption & Decryption \\ \hline
$1024$B & 125$\mu s$        & 193$\mu s$        \\ \hline
$4096$B & 141$\mu s$        & 220$\mu s$        \\ \hline
$16384$B & 220$\mu s$        & 367$\mu s$        \\ \hline
$65536$B & 519$\mu s$        & 702$\mu s$        \\ \hline
\end{tabular}
\end{table}

\begin{figure*}[!t]
\centering
  \subfigure[1 Consumer x 1 Producer]{\includegraphics[width=1.5in]{throughput_pk_1_1.png}\label{1a}}
  \hfil
  \subfigure[2 Consumers x 1 Producer]{\includegraphics[width=1.5in]{throughput_pk_2_1.png}\label{1b}}
  \hfil
  \subfigure[3 Consumers x 1 Producer]{\includegraphics[width=1.5in]{throughput_pk_3_1.png}\label{1c}}
  \hfil
  \subfigure[4 Consumers x 1 Producer]{\includegraphics[width=1.5in]{throughput_pk_4_1.png}\label{1d}}
\caption{Throughput per consumer in public key mode. 1 Producer N consumers}\label{exp1}
\end{figure*}

\begin{figure*}[!t]
\centering
  \subfigure[1 Consumer x 1 Producer]{\includegraphics[width=1.5in]{throughput_pk_1_1.png}\label{2a}}
  \hfil
  \subfigure[2 Consumers x 2 Producers]{\includegraphics[width=1.5in]{throughput_pk_2_2.png}\label{2b}}
  \hfil
  \subfigure[3 Consumers x 3 Producers]{\includegraphics[width=1.5in]{throughput_pk_3_3.png}\label{2c}}
  \hfil
  \subfigure[4 Consumers x 4 Producers]{\includegraphics[width=1.5in]{throughput_pk_4_4.png}\label{2d}}
\caption{Throughput per consumer in public key mode. N Producers N consumers}\label{exp2}
\end{figure*}




\begin{figure*}[!t]
\centering
  \subfigure[1 Consumer x 1 Producer]{\includegraphics[width=1.5in]{throughput_sk_1_1.png}\label{3a}}
  \hfil
  \subfigure[2 Consumers x 1 Producer]{\includegraphics[width=1.5in]{throughput_sk_2_1.png}\label{3b}}
  \hfil
  \subfigure[3 Consumers x 1 Producer]{\includegraphics[width=1.5in]{throughput_sk_3_1.png}\label{3c}}
  \hfil
  \subfigure[4 Consumers x 1 Producer]{\includegraphics[width=1.5in]{throughput_sk_4_1.png}\label{3d}}
\caption{Throughput per consumer in symmetric key mode. 1 Producer N consumers}\label{exp3}
\end{figure*}

\begin{figure*}[!t]
\centering
  \subfigure[1 Consumer x 1 Producer]{\includegraphics[width=1.5in]{throughput_sk_1_1.png}\label{4a}}
  \hfil
  \subfigure[2 Consumers x 2 Producers]{\includegraphics[width=1.5in]{throughput_sk_2_1.png}\label{4b}}
  \hfil
  \subfigure[3 Consumers x 3 Producers]{\includegraphics[width=1.5in]{throughput_sk_3_1.png}\label{4c}}
  \hfil
  \subfigure[4 Consumers x 4 Producers]{\includegraphics[width=1.5in]{throughput_sk_4_1.png}\label{4d}}
\caption{Throughput per consumer in symmetric key mode. N Producers N consumers}\label{exp4}
\end{figure*}


%\input{08-security}

\section{Conclusion}\label{conclusion}
\todo{TODO}

\ifCLASSOPTIONcaptionsoff
  \newpage
\fi

\tiny

\bibliographystyle{IEEEtran}
\bibliography{references}

% \begin{IEEEbiography}[{\includegraphics[width=1in,height=1.25in,clip,keepaspectratio]{picture}}]{John Doe}
% \blindtext
% \end{IEEEbiography}

\end{document}

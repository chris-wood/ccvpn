\section{Related Work} \label{sec:related}

There are generally two classes of related work: (1) anonymity networks, such
as ANDaNA \cite{dibenedetto2011andana} and AC3N \cite{tsudik2016ac3n}, and
(2) encryption-based access control techniques. ANDaNA was developed as a
proof-of-concept application-layer anonymizing network for NDN. It works by
creating single-use, ephemeral, and anonymizing circuits between a consumer
and producer. Each hop in this circuit uses onion decryption to decapsulate
interests and onion encryption to encrypt the result. A minimum of two hops are
needed to guarantee consumer and producer unlinkability. CCVPN seeks to address
a different threat: privacy instead of anonymity. Thus, only a single hop, which
creates a tunnel between a source and a sink AS, is needed in CCVPN. Moreover,
since these tunnels serve multiple traffic flows for \emph{all} consumers within
the same source, they are long-lived and persistent. This reduces the per-packet
and per-flow cryptographic operations needed to move packets between the source
and sink domains. AC3N \cite{tsudik2016ac3n} is an optimized version of ANDaNA
that keeps per-flow state in each hop of an anonymizing circuit to specify
ephemeral key identifiers for packets. This prevents intra-flow linkability
while simultaneously enabling symmetric-key encryption and decryption for more
efficient processing. However, since AC3N is still a per-consumer application,
anonymizing circuits cannot be shared among multiple geolocated consumers
in the same domain.

Content encryption seeks to solve the problem of data confidentiality rather than
privacy or anonymity. This technique allows content to be disseminated throughout
the network, since it cannot be decrypted by adversaries without the appropriate decryption key(s).
Many variations of this approach have been proposed based on general
group-based encryption \cite{Smetters2010}, broadcast encryption \cite{Misra2013,Ion2013}, and
proxy re-encryption \cite{Wood2014}. Kurihara et al. \cite{ifip15} generalized these specialized
approaches in a framework called CCN-AC, an encryption-based access control framework
that shows how to use manifests to explicitly specify and enforce other encryption-based
access control policies. Consumers use information in the manifest to (1) request appropriate
decryption keys and (2) use them to decrypt content object(s). The NDN NBAC \cite{yu2015name}
scheme is similar to \cite{ifip15} in that it allows decryption keys to be
flexibly specified by a data owner. However, it does this based on name engineering rules instead of
configuration. Interest-based access control \cite{ghali2015interest} is a different
type of access control scheme wherein content is optionally encrypted. Access
is protected by making the names of contents derivable only by authorized consumers.
NDN-ACE \cite{shangndn} is a recent access control framework for IoT environments
which includes a key exchange protocol for distributing secret keys to sensors.
All of these techniques, with few exceptions, use public-key cryptographic schemes
to protect only the \emph{payload} of content packets. They do not encapsulate complete
packets for private transmission between a source and sink domain. The exception is CCNxKE~\cite{ccnxke},
which specifies a key exchange protocol that can bootstrap secure sessions between a source
a sink. CCNxKE can be used by CCVPN to establish pairwise shared secrets between tunnel
endpoints, even though this step is not strictly required.

\section{Related Work} \label{sec:related}

\begin{itemize}
\item \cite{tsudik2016ac3n}
\item \cite{dibenedetto2011andana}
\item ...
\end{itemize}

Content-based encryption is arguably the most popular technique for protecting CCN content from
unauthorized disclosure. This technique permits content to be disseminated throughout
the network since it cannot be decrypted by adversaries without the appropriate decryption key(s).
Many variations of this approach have been proposed based on general
group-based encryption \cite{Smetters2010}, broadcast encryption \cite{Misra2013,Ion2013} and
proxy re-encryption \cite{Wood2014}. Kurihara et al. \cite{ifip15} generalized these specialized
approaches in a framework called CCN-AC, an encryption-based access control framework
that shows how to use manifests to explicitly specify and enforce other encryption-based
access control policies. Consumers use information in the manifest to (1) request appropriate
decryption keys and (2) use them to decrypt content object(s). The NDN NBAC \cite{yu2015name}
scheme is similar to \cite{ifip15} in that it allows decryption keys to be
flexibly specified by a data owner. However, it does this based on name engineering rules instead of
configuration. Interest-based access control \cite{ghali2015interest} is a different
type of access control scheme wherein content was optionally encrypted. Access
was protected by making the names of content derivable by only authorized consumers.
NDN-ACE \cite{shangndn} is a recent access control framework for IoT environments
which includes a key exchange protocol for distributing secret keys to sensors.
We revisit NDN-ACE in Section \ref{sec:compare}.
